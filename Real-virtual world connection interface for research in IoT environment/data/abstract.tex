% !TeX root = ../thuthesis-example.tex

% 中英文摘要和关键字

\begin{abstract}

近年来,计算设备的成本不断降低。我们即将进入以多个智能设备服务于用户为特征的AIoT时代。同时,通信技术快速发展,比如5G 和 6G网络,提供了高效连接这些设备的方法。因此,低功耗传感器和微计算控制器将在家电、衣服、家具等物体中联合地感知人们的行为并提供服务。然而,开发AIoT 应用和场景成本高昂。工程师们需要购买智能设备、实现连接,然后在此基础之上构建功能。在研究人员可以研究系统的用户体验之前,需要大量的工程开销。
为了解决这个限制,本文提出利用虚拟现实来模拟用户体验的方法。最近,虚拟现实技术得到了显著的发展,包括精确的手部识别技术,更轻的头盔和更令人满意的显示质量等。虚拟现实不仅在游戏中,同时在健康、教育和科学中都被证明是一个有效的工具。为了利用虚拟现实的优势来模拟用户体验,本文作者建立了一个虚实融合的智能空间交互研究平台。平台包括一个将真实设备和虚拟设备连接在一起的同步机制。换言之,设备的状态可以同时被虚拟世界和真实世界中的动作和事件来影响。本文描述了这个平台是如何开发的以及评测结果。

本文的贡献包括: 
\begin{enumerate}
    \item 创新提出了虚实融合AIoT开发和测试平台,用于开发物联网设备、场景和交互方法,可显著节省开发成本,提高效率。
    \item 提出虚实设备之间的数据同步机制,实现虚拟设备和物理设备的实时状态绑定,并支持设备之间互操作。
    \item 提供了NUIX Studio 开发工具包,具备丰富的可重用部件和功能,供研究人员快速开发AIoT 原型交互系统。
\end{enumerate}




  % 关键词用“英文逗号”分隔,输出时会自动处理为正确的分隔符
  \thusetup{
    keywords = {物联网, 虚拟现实, 人机交互},
  }
\end{abstract}

\begin{abstract*}

In recent years, the cost of computing devices is continuously decreasing and we will soon enter the AIoT era with a set of smart devices serving us around. Meanwhile, the rapid development of communication technology such as 5G- and 6G-networks provides an efficient way to connect these devices. As a result, energy-efficient sensors and microcontrollers in appliances, clothes, furniture, and so on are going to work together to sense people’s behavior and provide services. However, developing AIoT applications and scenarios is cost. Engineers have to purchase smart devices, make them connected and build functions on them. This requires a lot of engineering efforts before researchers can study the user experience of the system.   

To address this limitation, the thesis proposes to leverage Virtual Reality to simulate user experience.  Virtual Reality techniques have improved a lot recently, such as accurate hand recognition, lighter headsets, more satisfying display quality and so on. Virtual Reality has proven itself as an effective instrument not only for gaming purposes but for healthcare, education, and science. To leverage the advantage of Virtual Reality to simulate user experience, the author builds an interaction research platform for smart space based on Virtuality-Reality hybrid model. The platform includes a synchronization mechanism that keeps real and virtual devices connected. In other words, the status of the devices can be affected by both the actions and events in the virtual world and also those from real world. The thesis describes how such a platform is developed as well as the evaluation results.

The main contributions of this thesis include: 
  
  \begin{enumerate}
      \item A novel hybrid virtuality-reality model for developing IoT devices and scenarios, which can significantly save development costs and improve the efficiency of research. 
      \item A synchronization mechanism that binds the status of virtual devices and physical devices in real time and supports inter-operations. 
      \item A development kit, called NUIX-Studio, for researchers to fast prototype AIoT projects, with a rich set of reusable widgets and functionalities.
  \end{enumerate}


  
  % Use comma as seperator when inputting
  \thusetup{
    keywords* = {IoT, VR, HCI},
  }
\end{abstract*}
